\documentclass[./blockchain.tex
\graphicspath{{\subfix{./figures/}}}


\begin{document}
    \section{Bitcoin}
    \begin{frame}
        \frametitle{Bitcoin}

        \includegraphics[height=0.1\textheight]{figures/btc} \begin{huge}
                                                         Bitcoin
        \end{huge}

        \begin{itemize}
            \item Beginn: 2009, Satoshi Nakamoto
            \item BTC, Bitcoin-Symbol
            \item Das White-Paper: \href{https://bitcoin.org/bitcoin.pdf}{Bitcoin: A Peer-to-Peer Electronic Cash System}
            \item Anzahl Coins im Umlauf 18'923'415 (Stand: 8. Januar 2022)
            \item Maximale Anzahl Coins: 21'000'000
            \item Blockchain-Grösse 384,41 GB (Stand: 8. Januar 2022)
        \end{itemize}
    \end{frame}


    \begin{frame}
        \frametitle{Bitcoin}{Zahlungssystem}
        \begin{itemize}
            \item 2021 Tesla Motors kündete Zahlungsmöglichkeiten in Bitcoin an.
            \item USA, Canada und El Salvador unterhalten mehrere 10'000 Bitcoin-Bankomaten.
            \item \emph{Problem}: Eine Transaktion dauert 10 Minuten und kostet $\sim$ 20 USD.
            \item \emph{Lösung} Layer-2 Zahlungssystem Lightning Network
        \end{itemize}
    \end{frame}


    \begin{frame}
        \frametitle{Bitcoin}{Bitcoin als Anlage}

        \begin{itemize}
            \item Bitcoin gilt als elektronisches Gold
            \item Inflationssicherheit
            \item Mobilität (gegenüber physischen Anlagen)
            \item 2021/2022 Funds and ETFs investieren im grossen Stil
            \item 1 Bitcoin = 100 Mio Satoshi $\sim$ 40'000 USD
            \item Bspl. Trusts: Grayscale Bitcoin Trust hält 654'885 Bitcoins
        \end{itemize}
    \end{frame}

    \begin{frame}
        \frametitle{Bitcoin}{Trivia}

        \begin{description}
            \item[10'000 Bitcoins] war die erste kommerzielle Transaktion für zwei Pizzas im Jahr 2010.
            \item[Schätzungsweise 20\%] aller Bitcoins können nicht mehr verwendet werden. Die Private-Keys sind verloren wurden.
            \item[Im Jahr 2140] wird der letzte Bitcoin geprägt (mint) sein.
            \item[El Salvador] ist das erste Land, das Bitcoin als offizielles Zahlungmittel eingeführt hat.
        \end{description}


    \end{frame}


\end{document}

